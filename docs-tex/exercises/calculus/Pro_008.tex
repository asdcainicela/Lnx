\LnxPregunta{}{ 
	\begin{flushleft}  \Huge
		Dado  $A \in \mathcal{M}_2(\mathbb{R})$ es una matriz con dos autovalores reales  $\lambda_1 \neq \lambda_2$ y $\lambda_1 \lambda_2 > 0$.\\
		Evalue la integral,
	\end{flushleft}
	\vspace{1.5cm}
	$$
		I = \int_0^\infty \frac{\sin(Ax)}{x} dx
	$$
}

\LnxSolucion{Solución}{ 

Como $A$ es diagonalizable,

$$
A = P J_A P^{-1}, \quad \text{donde} \quad J_A = \begin{pmatrix} \lambda_1 & 0 \\ 0 & \lambda_2 \end{pmatrix}
$$

Aplicamos esta transformación a la integral,

$$
I = \int_0^\infty \frac{\sin(Ax)}{x} dx = P \left( \int_0^\infty \frac{\sin(J_A x)}{x} dx \right) P^{-1}
$$

Dado que $J_A$ es diagonal,

$$
\int_0^\infty \frac{\sin(J_A x)}{x} dx =
\begin{pmatrix}
\displaystyle \int_0^\infty \frac{\sin(\lambda_1 x)}{x} dx & 0 \\
0 &\displaystyle \int_0^\infty \frac{\sin(\lambda_2 x)}{x} dx
\end{pmatrix}
$$

Sabemos por Dirichlet’s integral que,

$$
\int_0^\infty \frac{\sin(\lambda x)}{x} dx = \operatorname{sign}(\lambda) \frac{\pi}{2}
$$

Por lo tanto,

$$
\int_0^\infty \frac{\sin(J_A x)}{x} dx =
\begin{pmatrix}
\operatorname{sign}(\lambda_1) \frac{\pi}{2} & 0 \\
0 & \operatorname{sign}(\lambda_2) \frac{\pi}{2}
\end{pmatrix}
$$
\newpage
Dado que $\lambda_1$ y $\lambda_2$ tienen el mismo signo,

$$
\int_0^\infty \frac{\sin(J_A x)}{x} dx =	\operatorname{sign}(\lambda_1) \frac{\pi}{2} I_2.
$$

Finalmente, aplicamos la transformación inversa,

$$
I = P \left( \operatorname{sign}(\lambda_1) \frac{\pi}{2} I_2 \right) P^{-1}
$$

$$
I = \operatorname{sign}(\lambda_1) \frac{\pi}{2}  P  I_2  P^{-1}
$$

Como la identidad conmuta con cualquier matriz,

$$
I = \operatorname{sign}(\lambda_1) \frac{\pi}{2} I_2
$$

Por lo tanto,   
}
\begin{LnxRptaBox}
	$$
	\int_0^\infty \frac{\sin(Ax)}{x} dx =
	\begin{cases}
		\frac{\pi}{2} I_2, & \text{si } \lambda_1, \lambda_2 > 0 \\
		&\\
		-\frac{\pi}{2} I_2, & \text{si } \lambda_1, \lambda_2 < 0
	\end{cases}
	$$
\end{LnxRptaBox} 
