\LnxPregunta{Pruebe  que,}{
	$$
		\displaystyle \int_0^1 x^2 \psi(x) \, dx = \ln\left(\dfrac{A^2}{\sqrt{2\pi}} \right)
	$$
}

\LnxSolucion{Solución}{

	Utilizando la descomposición fraccionaria parcial de $\psi(x)$,

	$$
		\psi(x) =  -\gamma +\lim_{n\to\infty} \pqty{ H_{n+1} - \frac{1}{x} - \sum_{k=1}^{n} \frac{1}{k+x} }
	$$

	Reemplazando en la integral,
	$$
		I= \lim_{n\to\infty} \int_{0}^{1} x^2 \left( -\gamma + H_{n+1} - \frac{1}{x} - \sum_{k=1}^{n} \frac{1}{k+x} \right) \dd{x}
	$$
	Integrando

	$$
		I= \lim_{n\to\infty} \int_{0}^{1} \left( \pqty{-\gamma + H_{n+1} }x^2- x - \sum_{k=1}^{n} \frac{x^2}{k+x} \right) \dd{x}
	$$

	$$
		I=\lim_{n\to\infty}  \left(  \eval{\pqty{-\gamma + H_{n+1} }\frac{x^3}{3}- \frac{x^2}{2}}_{0}^{1} - \sum_{k=1}^{n} \int_{0}^{1}\frac{x^2}{k+x} \dd{x} \right)
	$$

	Sea la integral, [Demostración para el lector]
	$$\displaystyle \int_{0}^{1}\frac{x^2}{k+x} \dd{x} = \frac{1}{2} - k + k^2 ( \ln(1 + k)-\ln(k) )
	$$
	Entonces,
	$$
		\sum_{k=1}^{n} \int_{0}^{1}\frac{x^2}{k+x} \dd{x}=\sum_{k=1}^{n} \pqty{ \frac{1}{2} - k + k^2 ( \ln(1 + k)-\ln(k) ) }
	$$

	$$
		\sum_{k=1}^{n} \int_{0}^{1}\frac{x^2}{k+x} \dd{x}=
		\frac{n}{2}-\frac{n(n+1)}{2}- \sum_{k=1}^{n} k^2 (\ln k - \ln(k+1))
	$$

	$$
		\sum_{k=1}^{n} \int_{0}^{1}\frac{x^2}{k+x} \dd{x}=
		-\frac{n^2}{2}- \sum_{k=1}^{n} k^2 (\ln k - \ln(k+1))
	$$

	%	$$
	%	\sum_{k=1}^{n} \int_{0}^{1}\frac{x^2}{k+x} \dd{x}=	-\pqty{\frac{n^2}{2}+ \sum_{k=1}^{n} (2k-1)\ln k + n^2 \ln(n+1) }
	%	$$

	Reemplazando en el límite,

	$$
		I=\lim_{n\to\infty} \left( \frac{H_{n+1}-\gamma}{3}  - \frac{1}{2}+ \frac{n^2}{2} + \sum_{k=1}^{n} k^2 (\ln k - \ln(k+1)) \right)
	$$

	$$
		I=\lim_{n\to\infty} \left( \frac{1}{3}\ln n  - \frac{1}{2} + \frac{n^2}{2}+ \sum_{k=1}^{n} (2k-1)\ln k - n^2 \ln(n+1) \right)
	$$


	Sean $(A_n)$ y $(B_n)$ denotados por,

	$$
		A_n = \frac{1^1 2^2 \cdots n^n}{n^{n^2/2+n/2+1/12}\; e^{-n^2/4}}
		\quad \text{y} \quad
		B_n = \frac{n!}{e^{-n}\; n^{n+1/2}}
	$$

	Sabemos que $A_n \rightarrow A$ y $B_n\rightarrow \sqrt{2\pi}$

	Podemos reescribir estas definiciones, mediante $\ln (.)$

	$$
		\sum_{k=1}^{n} k \ln k
		= \ln A_n + \left( \frac{n^2}{2} + \frac{n}{2} + \frac{1}{12} \right) \ln n - \frac{n^2}{4}
	$$

	$$
		\sum_{k=1}^{n} \ln k
		= \ln B_n + \left(n + \frac{1}{2}\right) \ln n - n.
	$$






	Reemplazando  $ \displaystyle \sum_{k=1}^{n} k \ln k$  \, y \,  $\displaystyle\sum_{k=1}^{n} \ln k $ en el límite

	$$
		I= \lim_{n\to\infty} \left(2\ln A_n - \ln B_n - n^2 \ln\left(1+\frac{1}{n}\right) + n - \frac{1}{2} \right)
	$$

	Tomando límite, tenemos

	$$
		I= 2\ln A - \ln\sqrt{2\pi}
	$$



}
\begin{LnxRptaBox}
	$$
		\int_0^1 x^2 \psi(x) \, dx = \ln\left(\dfrac{A^2}{\sqrt{2\pi}} \right)
	$$
\end{LnxRptaBox}
Donde $A$  es la constante de Glaisher-Kinkelin \\
