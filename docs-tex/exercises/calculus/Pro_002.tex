\LnxPregunta{Calcule el  limite,\\
\begin{equation*}
	\lim\limits_{x \to 0} 
	\frac{\displaystyle \int_{1}^{\infty} \frac{\ln^x(t)}{t^{2\gamma}} \dd{t}-(2\gamma-1)^{-x-1}}{e^x-1}
\end{equation*}\\
{\LARGE donde $\gamma$ es la constante de Euler-Mascheroni.}
}

\LnxSolucion{Solución}{
	
	Sea $a \in \mathbb{R}$, $f(x)$ y $g_1(x)$ funciones polinómicas, que cumplen $\deg[f(x)] < \deg[g_1(x)]$, suponemos que $g_1(x)$ se puede descomponer de la forma:
	\begin{equation}
		g_1(x) = (x - a)^n g(x)
	\end{equation}
	Y cumplen que $f(a) \neq 0$, $g(a) \neq 0$. Demostraremos que la razón
	\begin{equation}
		\frac{f(x)}{g_1(x)} = \frac{f(x)}{(x - a)^n g(x)}
	\end{equation}
	se puede expresar como:
	\begin{equation}
		\frac{f(x)}{g_1(x)} = \frac{A}{(x - a)^n} + \frac{h(x)}{(x - a)^{n-1} g(x)}
	\end{equation}
	Donde $A \in \mathbb{R}$ y además $h(x)$ es una función polinómica que cumple con $\deg[h(x)] < \deg[(x - a)^{n-1} g(x)]$.
	
	\textbf{Demostración:} Reescribimos:
	\begin{equation}
		\frac{f(x)}{g_1(x)} = \frac{A}{(x - a)^n} + \frac{f(x) - A g(x)}{(x - a)^n g(x)}
	\end{equation}
	Hacemos que el término $f(x) - A g(x)$ sea divisible por $x - a$. Para ello, evaluamos en $x = a$, lo que nos da:
	\begin{equation}
		f(a) - A g(a) = 0 \quad \Rightarrow \quad A = \frac{f(a)}{g(a)}
	\end{equation}
	Ya encontrado el valor de $A$, podemos decir que $f(x) - A g(x)$ se puede descomponer de la forma:
	\begin{equation}
		f(x) - A g(x) = (x - a) h(x)
	\end{equation}
	Sustituyendo en la ecuación original:
	\begin{equation}
		\frac{f(x)}{g_1(x)} = \frac{A}{(x - a)^n} + \frac{h(x)}{(x - a)^{n-1} g(x)}
	\end{equation}
	Y así queda demostrado.
	
	\textbf{Corolario:} Aplicando este resultado a $g_1(x) = (x - a)^2 g(x)$, obtenemos:
	\begin{equation}
		\frac{f(x)}{g_1(x)} = \frac{A}{(x - a)^2} + \frac{B}{x - a} + \frac{h(x)}{g(x)}
	\end{equation}
	Con:
	\begin{equation}
		A = \left.\frac{f(x)}{g(x)}\right|_{x=a}, \quad B = \left.\frac{d}{dx} \left( \frac{f(x)}{g(x)} \right)\right|_{x=a}
	\end{equation}
	
	Este resultado nos permite descomponer expresiones en fracciones parciales de manera eficiente.
	
}

Reemplazando los límites en L
$$
	L=\lim\limits_{x \to 0} \frac{1}{(2\gamma-1)^{x+1}} \cdot
\frac{\displaystyle \frac{ \Gamma(x+1)-1}{x}}{\displaystyle\frac{{e^x-1}}{x}}
$$

	$$
L= \frac{1}{(2\gamma-1)} \cdot
\frac{-\gamma}{1}
$$

Entonces el valor del límite es,

\begin{LnxRptaBox}
	$$
	 \lim\limits_{x \to 0} 
	 \frac{\displaystyle \int_{1}^{\infty} \frac{\ln^x(t)}{t^{2\gamma}} \dd{t}-(2\gamma-1)^{-x-1}}{e^x-1}=\frac{\gamma}{1-2\gamma} 
	$$
\end{LnxRptaBox}
