\LnxPregunta{ Calcule la serie si $\abs{x}<\frac{1}{4}$, }{
\vspace{1.5cm}
$$
	\sum_{k\ge0}\frac{(4k+1)!}{[(2k)!]^2}x^{2k}
$$
\vspace{0.1cm}
\begin{flushleft}
	{\Huge Propuesto por \textcolor{yellow}{Kav Maths}.  }
\end{flushleft}
}

\LnxSolucion{Solución}{

Sea $S(x)$ la serie y podemos reescribir,

$$
	S(x)=\sum_{k\ge0}\frac{(4k+1)!}{[(2k)!]^2}x^{2k} =\sum_{k\ge0}(4k+1)\frac{(4k)!}{[(2k)!]^2}x^{2k}
$$

%$$
%S(x)=\sum_{k\ge0}(4k+1)\binom{4k}{2k}x^{2k}
%$$ 

Distribuimos las series,

$$
	S(x)=2x\sum_{k\ge0}2k\frac{(4k)!}{[(2k)!]^2}x^{2k-1}+\sum_{k\ge0}\frac{(4k)!}{[(2k)!]^2}x^{2k}
$$

$$
	S(x)=2x \pdv{x}\pqty{\sum_{k\ge0}\frac{(4k)!}{[(2k)!]^2}x^{2k}}+\sum_{k\ge0} \frac{(4k)!}{[(2k)!]^2} x^{2k}
$$


%$$
%S(x)=2x\sum_{k\ge0}2k\binom{4k}{2k}x^{2k-1}+\sum_{k\ge0}\binom{4k}{2k}x^{2k}
%$$ 

%$$
%S(x)=2x \pdv{x}\pqty{\sum_{k\ge0}\binom{4k}{2k}x^{2k}}+\sum_{k\ge0}\binom{4k}{2k}x^{2k}
%$$ 


Sea $M(x)$ la siguiente serie,

$$
	M(x)=\sum_{k\ge0} \frac{(4k)!}{[(2k)!]^2}  x^{2k} = \sum_{k\ge0} \frac{2^{4k}}{(2k)!}\cdot\frac{(4k)!}{2^{4k}(2k)!}x^{2k}
$$

Simplificamos $ {(4k)!}/{ (2k)!} $
$$
	M(x)=\sum_{k\ge0} \frac{2^{4k}}{(2k)!}\cdot \frac{1\cdot3\cdots(4k-1)}{2^{4k} }x^{2k}
$$

$$
	M(x) = \sum_{k\ge0}  2^{4k}\cdot \frac{(-\tfrac{1}{2})(-\tfrac{3}{2})\cdots (-\frac{ (4k-1)}{2})}{2^{2k}(2k)!}x^{2k}
$$

$$
	M(x)=\sum_{k\ge0} 16^k\,\binom{-\tfrac{1}{2}}{2k} x^{2k}=\sum_{k\ge0}  \binom{-\tfrac{1}{2}}{2k} {(4x)}^{2k}
$$


Sea de forma general $H(a,x,r)$ donde \( |ax| < 1 \),

\[
	H(a,x,r)=\sum_{k\ge0}\binom{r}{2k}(ax)^{2k},
\]

Por la expansión binomial,
\[
	(1+ax)^r=\sum_{n\ge0}\binom{r}{n}(ax)^n
\]
Expresamos la suma en los términos de grado par e impar,
\[
	(1+ax)^r=\sum_{k\ge0}\binom{r}{2k}(ax)^{2k}+\sum_{k\ge0}\binom{r}{2k+1}(ax)^{2k+1}
\]

\[
	(1-ax)^r
	=\sum_{k\ge0}\binom{r}{2k}(ax)^{2k}-\sum_{k\ge0}\binom{r}{2k+1}(ax)^{2k+1}
\]

Al sumar las dos expresiones,
\[
	(1+ax)^r+(1-ax)^r=2\sum_{k\ge0}\binom{r}{2k}(ax)^{2k}
\]
Es decir $ H(a,x,r)$ es,
\[
	\sum_{k\ge0}\binom{r}{2k}(ax)^{2k}=\frac{(1+ax)^r+(1-ax)^r}{2}
\]


Entonces para $r=-\frac{1}{2}$ y $a=4$, tenemos que $M(x)$ es,

\newpage

$$
	M(x)=\frac{ (1-4x)^{-1/2}+(1+4x)^{-1/2}}{2}
$$

La derivada de $M(x)$ es,

$$
	M'(x)=(1-4x)^{-3/2}-(1+4x)^{-3/2}
$$

Además $S(x)$ en términos de $M(x)$ es,

$$
	S(x)=2x  M'(x) +M(x)
$$
De forma equivalente $M(x)$ y $2xM'(x)$ son,

$$
	M(x)=   \pqty{\frac{1}{2}-2x}(1-4x)^{-3/2}+\pqty{\frac{1}{2}+2x}(1+4x)^{-3/2}
$$

$$
	2xM'(x)=  2x(1-4x)^{-3/2}-2x(1+4x)^{-3/2}
$$

Sumamos $M(x)$ y $2xM'(x)$ para obtener $S(x)$


}
\begin{LnxRptaBox}
	$$
		S(x)=\frac{1}{2}\Bigl[(1-4x)^{-3/2}+(1+4x)^{-3/2}\Bigr],\quad  \abs{x}<\frac{1}{4}
	$$
\end{LnxRptaBox}

