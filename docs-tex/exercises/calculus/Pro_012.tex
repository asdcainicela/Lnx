\LnxPregunta{ Calcule la integral, }{  
	\vspace{1.5cm}
	$$
	 \int_0^1 \frac{\ln^3(x+1)}{(x+1)(x+3)} \dd{x}
	$$
	\vspace{0.1cm}
	\begin{flushleft}
	%	{\Huge Si $ \text{Li}_3 $ es la función trilogarítmica. }\\
		{\Huge Propuesto por \textcolor{yellow}{MathGauss}.  }
	\end{flushleft}
}

\LnxSolucion{Solución}{ 

Podemos reescribir,

$$
I = \frac{1}{2}\int_0^1  \frac{[(x+3)-(x+1)]\ln^3(x+1)}{(x+1)(x+3)} \dd{x}
$$

La integral se convierte en,
\[
I = \underbrace{\frac{1}{2}\int_0^1 \frac{\ln^3(x+1)}{x+1} \dd{x} }_{I_1}- \underbrace{\frac{1}{2}\int_0^1 \frac{\ln^3(x+1)}{x+3} \dd{x}}_{I_2}  =I_1-I_2
\]

La integral $I_1$ es inmediata,

$$
I_1=\frac{1}{2}\int_0^1 \frac{\ln^3(x+1)}{x+1} \dd{x} =\eval{\frac{\ln^4(x+1)}{8}}_{0}^{1}=\frac{\ln^4(2)}{8}
$$



En  $I_2$, hacemos $t = x+1$, $\dd{t} = \dd{x}$, además $\frac{1}{t+2} = \frac{1}{2}\sum_{k=0}^\infty \left(-\frac{t}{2}\right)^k$,

\[
I_2 = \frac{1}{2}\int_1^2 \frac{\ln^3 t}{t+2} \dd{t} =\frac{1}{4}\sum_{k=0}^\infty \frac{(-1)^k}{2^k} \underbrace{\int_{1}^{2} t^k \ln^3(t) \dd{t}}_{I3}
\]

Para $I_3$ usamos la siguiente integral recursiva, [demostración para el lector - Use integración por partes]
$$
\int x^m \ln^n(x)\dd{x} = \frac{x^{m+1} \ln^n(x)}{m+1} -\frac{n}{m+1} \int x^m \ln^{n-1}(x)\dd{x}
$$

O de manera equivalente, [demostración para el lector]

\[
\int x^m \ln^n(x) \, dx = \frac{x^{m+1}}{m+1} \sum_{i=0}^{n} \frac{(-1)^i \, n!\, \ln^{n-i}(x)}{(n-i)! \,(m+1)^i} + C
\]

Entonces para $I_3$ , tenemos que $n=3$\quad y $m=k$,

$$
I_3=\int_1^2 t^k \ln^3 t \dd{t}=\eval{\frac{6t^{k+1}}{k+1} \sum_{i=0}^{3} \frac{(-1)^i \, \ln^{3-i}(t)}{(3-i)! \,(k+1)^i} }_{1}^{2}
$$
\[
I_3=  \eval{t^{k+1}\left[\frac{\ln^3 t}{k+1} - \frac{3\ln^2 t}{(k+1)^2} + \frac{6\ln t}{(k+1)^3} - \frac{6}{(k+1)^4}\right]}_{1}^{2}
\] 

 \begin{equation*}
	\scalebox{0.9}{
		$\displaystyle 
		I_3= 2^{k+1}\left[\frac{(\ln 2)^3}{k+1} - \frac{3(\ln 2)^2}{(k+1)^2} + \frac{6\ln 2}{(k+1)^3} - \frac{6}{(k+1)^4}\right] + \frac{6}{(k+1)^4}
		$}
\end{equation*}

 

Obtenemos $I_2$  sustituyendo en la serie a $I_3$, $I_2 = \frac{1}{4}\sum_{k=0}^\infty \frac{(-1)^k}{2^k} I_3$
 
\begin{equation*}
	\scalebox{0.8}{$\displaystyle	I_2 = \frac{1}{4}\sum_{k=0}^\infty \frac{(-1)^k}{2^k}\pqty{  2^{k+1}\left[\frac{(\ln 2)^3}{k+1} - \frac{3(\ln 2)^2}{(k+1)^2} + \frac{6\ln 2}{(k+1)^3} - \frac{6}{(k+1)^4}\right] + \frac{6}{(k+1)^4} } $}
\end{equation*} 
\begin{equation*}
	\scalebox{0.77}{$\displaystyle	I_2 = \frac{1}{2}\sum_{k=0}^\infty (-1)^k\left[\frac{(\ln 2)^3}{k+1} - \frac{3(\ln 2)^2}{(k+1)^2} + \frac{6\ln 2}{(k+1)^3} - \frac{6}{(k+1)^4}\right] + 3\sum_{k=0}^\infty \frac{(-1)^k}{2^{k+1}(k+1)^4} $}
\end{equation*}
 

Reconocemos las siguientes series,
 $$\sum_{k=0}^\infty \frac{(-1)^k}{k+1}  = \ln(2)$$ 
 $$ \eta(s) = \sum_{n=1}^{\infty} \frac{(-1)^{n-1}}{n^s}, \quad  \text{ (función eta)} $$ 
 Y la función Polilogaritmo $\Li_s(z) = \sum_{k=1}^\infty \frac{z^k}{k^s},\quad \abs{z}<1$,
 $$
 \sum_{k=0}^\infty \frac{(-1)^k}{2^{k+1}(k+1)^4}  =
  -\sum_{k=1}^\infty \frac{\pqty{- \frac{1}{2}}^{k} }{k^4}  =-\Li_4\left(-\frac{1}{2}\right) 
  $$
En $I_2$, \quad  $\eta(s) = (1 - 2^{1-s}) \zeta(s)$,\quad $\zeta(2)=\frac{\pi^2}{6}, \zeta(4)=\frac{\pi^4}{90}$%  [Restringir s]
 
\begin{equation*}
	\scalebox{0.8}{$\displaystyle 
		I_2=\frac{1}{2}\bqty{\ln^3(2)\ln(2)- 3\ln^2(2)\eta(2)+ 6\ln(2)\eta(3)-6\eta(4)}+3\bqty{ -\Li_4\left(-\frac{1}{2}\right) }
		$}
\end{equation*}

\begin{equation*}
\scalebox{0.9}{
	$\displaystyle 
I_2=\frac{1}{2}\bqty{\ln^4(2)- 3\ln^2(2)\frac{\zeta(2)}{2}+ 6\ln(2)\frac{3\zeta(3)}{4}-6\frac{7\zeta(4)}{8}}-3\Li_4\left(-\frac{1}{2}\right) 
	$}
\end{equation*}
 
%También $\zeta(2)=\frac{\pi^2}{6}, \zeta(4)=\frac{\pi^4}{90}$. Reemplazamos en $I_2$,
\[
I_2 = \frac{\ln^4(2)}{2} - \frac{\pi^2\ln^2(2)}{8} + \frac{9\ln (2)\,\zeta(3)}{4} - \frac{7\pi^4}{240} - 3\Li_4\left(-\frac{1}{2}\right)
\]

 La integral original es $I=I_1-I_2$,
 
 \begin{equation*}
 	\scalebox{0.7}{
 		$\displaystyle 
 		I = I_1 - I_2 = \frac{\ln^4 (2)}{8} -  \bqty{ \frac{\ln^4(2)}{2} - \frac{\pi^2\ln^2(2)}{8} + \frac{9\ln (2)\,\zeta(3)}{4} - \frac{7\pi^4}{240} - 3\Li_4\left(-\frac{1}{2}\right) }
 		$}
 \end{equation*}
  

Simplificamos, 


}
\begin{LnxRptaBox} 
	 \begin{equation*}
		\scalebox{0.9}{
			$\displaystyle 
			 I = -\frac{3}{8}\ln^4 (2 )+ \frac{\pi^2}{8}\ln^2 (2) - \frac{9}{4}\zeta(3)\ln (2) + \frac{7\pi^4}{240} + 3\Li_4\left(-\frac{1}{2}\right)
			$}
	\end{equation*} 
\end{LnxRptaBox} 
Donde $\zeta(3)$ es la constante de Apéry, $\Li_4(\cdot)$ es la función Polilogaritmo.
