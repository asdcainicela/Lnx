\LnxPregunta{  Calcule  la integral, }{
\vspace{1.5cm}
\[
	\int_0^1 x^m \left\{ \frac{1}{x} \right\}^b \ln^n{x} \, dx
\]
\vspace{0.1cm}
\begin{flushleft}
	%	{\Huge Si $ \text{Li}_3 $ es la función trilogarítmica. }\\
	{\Huge	donde \( m \geq b \), \( m, b \in \mathbb{Z}^+ \) y \( n \in \mathbb{Z}^{+} \cup \{0\} \)  \\
		además, $ \{y\} = y - \lfloor y \rfloor$. }
\end{flushleft}
}

\LnxSolucion{Solución}{

vemos que
$$
	I(m,b,n)=	\int_0^1 x^m \left\{ \frac{1}{x} \right\}^b \ln^n{x} \, dx = \left. \frac{\partial^n}{\partial a^n}  \left( \int_0^1 x^a \left\{ \frac{1}{x} \right\}^b dx \right)\right|_{a=m}
$$

Calculamos $I(a,b,0)$, \( \{x\} = x - \lfloor x \rfloor \), hacemos \( x = 1/y \),


\[
	I(a,b,0)=\int_0^1 x^a \left\{ \frac{1}{x} \right\}^b dx = \int_1^\infty y^{-a-2} \{y\}^b dy
\]
Dividimos la integral en intervalos \( [k, k+1] \):
\[
	= \sum_{k=1}^\infty \int_k^{k+1} y^{-a-2} \{y\}^b dy = \sum_{k=1}^\infty \int_k^{k+1} y^{-a-2} (y - \lfloor y \rfloor)^b dy
\]
Como \( y \in [k, k+1] \), reemplazamos \( \lfloor y \rfloor \) por \( k \):
\[
	= \sum_{k=1}^\infty \int_k^{k+1} y^{-a-2} (y - k)^b dy \equiv I_{a,b,k}
\]
donde denotamos la integral por \( I_{a,b,k} \). Aplicando integraci\'on por partes,

\[
	= -\left. \frac{y^{-a-1}}{a + 1} (y - k)^b \right|_{y=k}^{y=k+1} + \frac{b}{a + 1} \underbrace{ \int_k^{k+1} y^{-a-1} (y - k)^{b-1} dy}_{I_{a-1,b-1,k}}.
\]
Multiplicando por \( a + 1 \),
\begin{equation}
	(a + 1) I_{a,b,k} = -\frac{1}{(k + 1)^{a+1}} + b I_{a-1,b-1,k}
	\label{eq1}
\end{equation}



*{Caso \( b = 1 \)},
\[
	\int_0^1 x^a \left\{ \frac{1}{x} \right\} dx = \int_1^\infty \frac{\{y\}}{y^{a+2}} dy
\]
Dividiendo en intervalos:
\[
	= \sum_{k=1}^\infty \int_k^{k+1} \frac{y - k}{y^{a+2}} dy = \sum_{k=1}^\infty \left( \int_k^{k+1} \frac{1}{y^{a+1}} dy - \int_k^{k+1} \frac{k}{y^{a+2}} dy \right)
\]
\[
	= \frac{1}{a} \sum_{k=1}^\infty \left( \frac{1}{k^a} - \frac{1}{(k + 1)^a} \right) - \frac{1}{a + 1} \sum_{k=1}^\infty \left( \frac{1}{k^a} - \frac{k}{(k + 1)^{a+1}} \right)
\]
\begin{equation}
	\int_0^1 x^a \left\{ \frac{1}{x} \right\} dx 	= \frac{1}{a} - \frac{\zeta(a + 1)}{a + 1}
	\label{eq5}
\end{equation}


*{Caso  \( a = b = p \)},
\begin{equation}
	(p + 1) I_{p,p,k} - p I_{p-1,p-1,k} = -\frac{1}{(k + 1)^{p+1}}
	\label{eq2}
\end{equation}

\begin{equation}
	I_{p,p,k} = \frac{1}{k(k + 1)(p + 1)} - \frac{1}{p + 1} \sum_{i=1}^p \frac{1}{(k + 1)^{i+1}}
	\label{eq3}
\end{equation}

Usando esto en la integral original:
\[
	\int_0^1 x^p \left\{ \frac{1}{x} \right\}^p dx = \sum_{k=1}^\infty I_{p,p,k}
\]

\[ = \sum_{k=1}^\infty \left( \frac{1}{k(k + 1)(p + 1)} - \frac{1}{p + 1} \sum_{i=1}^p \frac{1}{(k + 1)^{i+1}} \right)
\]
$$
	= \frac{1}{p + 1} \sum_{k=1}^\infty \frac{1}{k(k + 1)} - \frac{1}{p + 1} \sum_{i=1}^p \left( \zeta(i + 1) - 1 \right)
$$
\begin{equation}
	\int_0^1 x^p \left\{ \frac{1}{x} \right\}^p dx  = 1 - \frac{1}{p + 1} \sum_{i=1}^p \zeta(i + 1)
	\label{eq4}
\end{equation}



*{Caso  \( a > b > 1 \)}
Multiplicando ambos lados de \eqref{eq1} por \( a!/b! \),
\[
	\frac{(a + 1)!}{b!} I_{a,b,k} = -\frac{a!}{b!(k + 1)^{a+1}} + \frac{a!}{(b - 1)!} I_{a-1,b-1,k}
\]
Iterando esta relaci\'on,

\begin{equation*}
	\scalebox{0.8}{$\displaystyle
			\frac{(a + 1)!}{b!} I_{a,b,k} = (a - b + 2)! I_{a-b+1,1,k} - \sum_{i=1}^{b-1} \frac{(a - b + i + 1)!}{(i + 1)! (k + 1)^{a - b + i + 2}}
		$}
\end{equation*}

Sumando sobre \( k \geq 1 \) y usando la definici\'on de \( I_{a,b,k} \):

\begin{equation*}
	\scalebox{0.7}{$\displaystyle
			\frac{(a + 1)!}{b!} \int_0^1 x^a \left\{ \frac{1}{x} \right\}^b dx = (a - b + 2)! \int_0^1 x^{a - b + 1} \left\{ \frac{1}{x} \right\} dx - \sum_{k=1}^\infty \sum_{i=1}^{b-1} \frac{(a - b + i + 1)!}{(i + 1)! (k + 1)^{a - b + i + 2}}
		$}
\end{equation*}


Usando \eqref{eq5} y simplificando,

\begin{equation}
	\scalebox{0.55}{$\displaystyle
			\int_0^1 x^a \left\{ \frac{1}{x} \right\}^b dx = \frac{(a - b + 2)! b!}{(a - b + 1)(a + 1)!} - \frac{(a - b + 1)! b!}{(a + 1)!} \zeta(a - b + 2)  - \frac{b!}{(a + 1)!} \sum_{j=1}^{b-1} \frac{(a - j + 1)!}{(b - j + 1)!} \left( \zeta(a - j + 2) - 1 \right)
		$}
	\label{eq8}
\end{equation}


Finalmente, combinando los resultados de (\ref{eq4}), (\ref{eq5}) y (\ref{eq8}), obtenemos para \( a, b \) enteros positivos, \( a \geq b \), que

\begin{equation*}
	\scalebox{0.9}{$\displaystyle
			I(a,b,0) =\frac{b!}{(a + 1)!} \left( (a - b)! - \sum_{j=1}^b \frac{(a - j + 1)!}{(b - j + 1)!} \left( \zeta(a - j + 2) - 1 \right) \right)
		$}
\end{equation*}

Además
$$
	I(m,b,n)=	  \left. \frac{\partial^n}{\partial a^n}  \left( \int_0^1 x^a \left\{ \frac{1}{x} \right\}^b dx \right)\right|_{a=m}
$$




}
\begin{LnxRptaBox}
	\begin{equation*}
		\scalebox{0.6}{$\displaystyle
			\int_0^1 x^m \left\{ \frac{1}{x} \right\}^b \ln^n{x} \, dx = \left. \frac{\partial^n}{\partial a^n} \left( \frac{b!}{(a + 1)!} \left( (a - b)! - \sum_{j=1}^b \frac{(a - j + 1)!}{(b - j + 1)!} \left( \zeta(a - j + 2) - 1 \right) \right) \right)  \right|_{a=m}
		$}
	\end{equation*}
\end{LnxRptaBox}
donde \( \zeta(.) \) es la función zeta de Riemann
.\\


\begin{itemize}
	\item Ejemplo1, del caso  \eqref{eq5}

	      $$
		      \int_0^1 x^2 \left\{ \frac{1}{x} \right\} \ln^2(x) \, dx  =\left. \frac{\partial^2 }{\partial s^2} \right|_{s=2} \left( \int_0^1 x^s \left\{ \frac{1}{x} \right\}  \, dx  \right)
	      $$

	      \[
		      =  \left.  \frac{d^2}{ds^2} \right|_{s=2} \left( \frac{1}{s} - \frac{\zeta(s+1)}{s+1} \right)
	      \]

	      \[
		      =\left.  \frac{2}{s^3} - \frac{(s+1)^2 \zeta''(s+1) - 2(s+1)\zeta'(s+1) + 2\zeta(s+1)}{(s+1)^3} \right|_{s=2}
	      \]

	      \[
		      \int_0^1 x^2 \left\{ \frac{1}{x} \right\} \ln^2{x} \, dx = \frac{1}{4} - \frac{2}{27}\zeta(3) + \frac{2}{9}\zeta'(3) - \frac{1}{3}\zeta''(3)
	      \]



\end{itemize}


