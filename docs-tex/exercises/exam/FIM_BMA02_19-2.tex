


\LnxExam{
	\frenchspacing
	\begin{flushleft}
		\begin{minipage}{2.3cm}
			\begin{center}
				\includegraphics[width=0.9\linewidth]{img/logo_uni.png}
			\end{center}
		\end{minipage}
		\begin{minipage}{11cm}
			\begin{flushleft}
				{
					\Large \textsc{{\bf UNIVERSIDAD NACIONAL DE INGENIERÍA}	}		\\
					\mbox{{\bf Facultad de Ingeniería Mecánica}} \\
				}
			\end{flushleft}
		\end{minipage}
		\begin{minipage}{2.3cm} 			
			%\LnxSize{1} % Logo escalado
			\includegraphics[width=1\linewidth]{img/logo_light_edit.png}
		\end{minipage}
	\end{flushleft}
	%%FIN
	
	\vspace{15pt}
	\begin{center}
		{\bf {\Large Segunda práctica calificada Cálculo Integral (BMA02) }}\\\vspace{.5 cm}
		{\bf {\large 2019-II}}
	\end{center}
	%fin del logo
	
	%enumeracion
	Tiempo: 3 días 
	\begin{enumerate} \item 
		{\bf Hallar los siguientes límites}
		\vspace{10pt}
		
			a) 
			$
			\displaystyle \lim_{n \rightarrow \infty } \frac{\sqrt[n]{\pqty{1^2+n^2}(2^2+n^2)\ldots ((2n)^2+n^2))}}{n^4}
			$
			\\
			b)
			$
			\displaystyle \lim_{x \rightarrow 0^{+} } \int_0^1 \frac{\ln ( x^{x^2}\sen(x \sqrt{t})+x^{x^2} )}{x}\, \mathrm{d}t
			$
		\item 
		{\bf Sea $f: \mathbb{R} \rightarrow \mathbb{R} $ una función continua. Mostrar } %Descriptor personalizado 
		\vspace{10pt}
		
		a) 
		$
		\displaystyle \int_{1}^{3} \cos (x^2-4x)(x+f(\sen (x^2-4x))(x-2)-2)=0
		$
		\\
		
		b)
		$
		\displaystyle \int_{0}^{\pi} \frac{sen(x)f(x)}{f(x)+f(\pi-x)}=1
		$
		
		\vspace{10pt}
		
		\item %problema 3
		{\bf Dado  $n \in \mathbb{N} $, a y b números positivos, hallar los siguientes integrales  } 
		\vspace{10pt}
		\begin{multicols}{2}
			
			a) 
			$
			\displaystyle \int_{0}^{\pi} \frac{x \sen^{2n+2}x \cos^2 x }{  \sen^{2n}x +\cos^{2n}x} \mathrm{d} x
			$
			\\
			b)
			$
			\displaystyle \int_{0}^{ \arctan(\frac{a}{b}) } \frac{b\sen x +a\cos x}{a\sen x+b\cos x} \, \mathrm{d} x  
			$
		\end{multicols} % Final del primer problema
		\item %problema 4
		{\bf Sea $k \in \mathbb{R } \hspace{.25 cm}\mbox{y} \hspace{.25cm} f: \mathbb{R} \rightarrow \mathbb{R} $ definidas respectivamente como}
		
		\begin{center}
			$f(x):=\displaystyle \int_{0}^{ x } e^{t^2} \, \mathrm{d}t $ , \hspace{1 cm} $k:=f(1) $
		\end{center}
		{\bf Hallar las siguientes integrales en términos de k}
		\begin{multicols}{2}
			a) 
			$
			\displaystyle \int_{0}^{ k } \left[ \frac{y}{e^{(f^{-1} (y))^2}} -f^{-1} (y) \right]\, \mathrm{d} y
			$
			\\
			b)
			$
			\displaystyle \int_{0}^{ 1 } \int_{0}^{ z } \int_{0}^{ y }  ze^{x^2+y^2}  \mathrm{d} x \  \mathrm{d} y \  \mathrm{d} z
			$
		\end{multicols} % Final del primer problema
		
	\end{enumerate}
	
} 

\newpage
\SetLnxColors{LnxblackF}{white}
\LnxSolucion{Solución}{

\solucion{Solución 1.a} 

Reescribiendo el límite,

$$
\Omega= \lim_{n \to \infty } \frac{\sqrt[n]{ \prod_{i = 1 }^{2n} (i^2 + n^2) }}{n^4} 
$$

Tomando el logaritmo natural en ambos lados,

$$
\ln (\Omega)= \lim_{n \to \infty } \ln \pqty{\prod_{ i =1 }^{2n} (i^2 + n^2)}^{1/n}-2\ln(n^2)  
$$

Se sabe que $   \ln (b)^a=a\ln(b)$

$$
\ln (\Omega)= \lim_{n \to \infty } \frac{\ln \left(    \prod_{ i =1 }^{2n} (i^2 + n^2)     \right)-2n\ln(n^2)}{n}  
$$

Además, por propiedad $ \ln(\prod_{i=1}^{n}a_i)=\sum_{i=1}^{n}\ln(a_i)$

$$
\ln (\Omega)= \lim_{n \to \infty } \frac{\pqty{\sum_{ i =1 }^{2n} \ln(i^2 + n^2)}- \sum_{i=1}^{2n}\ln(n^2)}{n}  
$$

También, $ \sum_{i=1}^{k} y_i-z_i = \sum_{i=1}^{k}y_i - \sum_{i=1}^{k} z_i $ 

$$
\ln (\Omega)=  \lim_{n \to \infty } \frac{     \sum_{ i =1 }^{2n} \left[ \ln(i^2 + n^2)   -  \ln(n^2)\right]  }{n} 
$$


$$
 \ln (\Omega) =
  \lim_{n \to \infty } \sum_{i=1}^{2n} \ln\left( \frac{i^2+n^2}{n   ^2}\right) \frac{1}{n}
$$

$$
\ln (\Omega)=  \lim_{n \to \infty }  \sum_{ i =1 }^{2n} \ln\left( \left( \frac{i}{n} \right) ^2+1\right)    \frac{1}{n} 
$$

Mediante sustitución $2n=p$

$$
\ln (\Omega)= \lim_{p \to \infty }  2\sum_{ i =1 }^{p} \ln\left( 4\left( \frac{i}{p} \right) ^2+1\right)   \frac{1}{p}
$$

\begin{definicion}
	La \textbf{integral definida de Riemann} de una función \( f(x) \) en \([a,b]\) se define como el límite de la suma de Riemann:
	\[
	\int_a^b f(x) \,dx = \lim_{n \to \infty} \sum_{i=1}^{n} f(x_i^*) \Delta x,
	\]
	donde \( x_i^* \) es un punto dentro de cada subintervalo y \( \Delta x = \frac{b-a}{n} \).
\end{definicion}

En consecuencia, tenemos

$$
\ln (\Omega) =
2\ \int_0^1 \ln(4x^2+1)\dd{x}
$$

Resolviendo la integral

$$
\ln (\Omega)=  \eval{ 2x\ln(4x^2+1)-4x+2\tan^{-1}(2x)}_0^1 
$$

$$
\ln (\Omega)= 2\ln(5)+2\arctan(2)-4 $$

\begin{LnxRptaBox}
	$$
 \Omega=25   e^{2\arctan(2)-4}
	$$
\end{LnxRptaBox}

\solucion{Solución 1.b} 

Se debe encontrar:


$$
\omega = \lim_{x \to 0^{+} }\pqty{ \int_0^1 \frac{\ln ( x^{x^2}\sen(x \sqrt{t})+x^{x^2} )}{x} \dd{t}}
$$

$$
\omega = \lim_{x \to 0^{+} } \int_0^1 x\ln (x) \dd{t}+\int_0^1 \frac{\ln ( \sen(x \sqrt{t})+1 )}{x} \dd{t}
$$

$$ 
\omega = \lim_{x \to 0^{+} } \pqty{x\ln (x)+\int_0^1 \frac{\ln ( \sen(x \sqrt{t})+1 )}{x} \dd{t}}
$$

Aplicamos la regla de L’Hôpital 

$$
\omega = \lim_{x \to  0^{+} } \pqty{ \frac{\dv{x}(\ln(x))}{\dv{x}(1/x)}+\int_0^1 \frac{\pdv{x}\ln ( \sen(x \sqrt{t})+1 )}{\pdv{x}x} \dd{t}}
$$

Observamos que operando obtenemos,

$$
\omega = \lim_{x \to 0^{+}} \pqty{-x +\int_0^1 \frac{\sqrt{t} \cos (x\sqrt{t}) }{\sen (x\sqrt{t}) +1}\dd{t}} 
$$

Mediante el límite y operando la integral
$$
\omega = \int_0^1 \sqrt{t} \dd{t}= \frac{2}{3}
$$



\begin{LnxRptaBox}
	$$
	\lim_{x \to 0^{+} }\pqty{ \int_0^1 \frac{\ln ( x^{x^2}\sen(x \sqrt{t})+x^{x^2} )}{x} \dd{t}} = \frac{2}{3}
	$$
\end{LnxRptaBox}


\solucion{Solución 2.a} 

Dada la integral

$$
 \int_1^3 \cos(x^2-4x)(x+f(\sen(x^2-4x))(x-2)-2)  
$$
 
Realizando un cambio de variable $ x-2=u$,

$$
 \int_{-1}^1 u\ \cos(u^2-4)(1+f(\sen(u^2-4)) 
$$

$$
g(u)=u\ \cos(u^2-4)(1+f(\sen(u^2-4)) 
$$
 $$
 g(-u)=-u\ \cos(u^2-4)(1+f(\sen(u^2-4)) 
 $$
 Vemos que  $g(u)$ es impar,
 
 $$
 g(u)=-g(u)
 $$
 
 \begin{proposicion}
 	Si \( g: [-b,b] \to \mathbb{R} \) es una función impar, entonces se cumple que:
 	\[
 	\int_{-b}^{b} g(u) \,du = 0.
 	\]
 \end{proposicion}
 
 Por la proposición
\begin{LnxRptaBox}
	$$
	 \int_{-1}^1 u\ \cos(u^2-4)(1+f(\sen(u^2-4)) = 0 
	$$
\end{LnxRptaBox}

\solucion{Solución 2.b} 

\begin{equation}
I=  \int_{0}^{\pi} \frac{sen(x)f(x)}{f(x)+f(\pi-x)}	
\label{eq:fim_19_2_1}
\end{equation}

Usamos la siguiente proposición,
\begin{proposicion}
	Sea \( g: [0,a] \to \mathbb{R} \) una función integrable en \([0,a]\). Entonces, se cumple la siguiente propiedad de simetría,
	\[
	\int_0^a g(x) \,dx = \int_0^a g(a-x) \,dx.
	\]
\end{proposicion}

Por lo tanto,

$$
I=  \int_{0}^{\pi} \frac{sen(\pi-x)f(\pi-x)}{f(\pi-x)+f(\pi-(\pi-x))}
$$

Simplificando  $I$,

\begin{equation}
I=  \int_{0}^{\pi} \frac{sen(x)f(\pi-x)}{f(\pi-x)+f(x)}
\label{eq:fim_19_2_2}
\end{equation}

Sumando las ecuaciones (\ref{eq:fim_19_2_1}) y  (\ref{eq:fim_19_2_2}) se tiene,

$$
2I=  \int_{0}^{\pi} \frac{sen(x)\pqty{f(x)+f(\pi-x)} }{f(x)+f(\pi-x)}
$$

Simplificando
$$
2I=  \int_{0}^{\pi} sen(x)=2
$$

\begin{LnxRptaBox}
	$$
	I= \int_{0}^{\pi} \frac{sen(x)f(x)}{f(x)+f(\pi-x)}	=  1
	$$
\end{LnxRptaBox}

\solucion{Solución 3.a} 
\begin{equation}
	I=  \int_{0}^{\pi} \frac{x \sen^{2n+2}(x) \cos^2 (x) }{  \sen^{2n}(x) +\cos^{2n}(x)} \dd{x}
	\label{eq:fim_19_2_3}
\end{equation}

De la proposición 2, tenemos

$$
I=  \int_{0}^{\pi} \frac{(\pi-x) \sen^{2n+2}(\pi-x) \cos^2 (\pi-x) }{  \sen^{2n}(\pi-x) +\cos^{2n}(\pi-x)} \dd{x}
$$

Simplificando,

\begin{equation}
	I=  \int_{0}^{\pi} \frac{(\pi-x) \sen^{2n+2}(x) \cos^2 (x) }{  \sen^{2n}(x) +\cos^{2n}(x)}\dd{x}
	\label{eq:fim_19_2_4}
\end{equation}

Sumando las ecuaciones (\ref{eq:fim_19_2_3}) y  (\ref{eq:fim_19_2_4}) se tiene,

$$
2I=  \int_{0}^{\pi} \frac{\pi \sen^{2n+2}(x) \cos^2 (x) }{  \sen^{2n}(x) +\cos^{2n}(x)} \dd{x}
$$

$$
2I= \int_{-\frac{\pi}{2}}^{\frac{\pi}{2}} \frac{\pi \cos^{2n+2}(x) \sen^2(x)}{\sen^{2n}(x)+\cos^{2n}(x)} \dd{x}
$$
\begin{proposicion}
	Si \( f: [-a,a] \to \mathbb{R} \) es una función par,  entonces se cumple la siguiente propiedad,
	\[
	\int_{-a}^{a} f(x) \,dx = 2 \int_{0}^{a} f(x) \,dx.
	\]
\end{proposicion}

De la proposición, tenemos que,

$$
 2I=   2\int_{0}^{\frac{\pi}{2}} \frac{\pi \cos^{2n+2}x \sen^2 x }{  \sen^{2n}x +\cos^{2n}x} \dd{x}
$$

\begin{equation}
	I=  \int_{0}^{\frac{\pi}{2}} \frac{\pi \cos^{2n+2}x \sen^2 x }{  \sen^{2n}x +\cos^{2n}x} \dd{x}
	\label{eq:fim_19_2_5}
\end{equation}

Usamos la primera proposición y simplificando


\begin{equation}
	I=  \int_{0}^{\frac{\pi}{2}} \frac{\pi \sen^{2n+2}x \cos^2 x }{  \sen^{2n}x +\cos^{2n}x} \dd{x
	}
	\label{eq:fim_19_2_6}
\end{equation}

Sumando las ecuaciones (\ref{eq:fim_19_2_5}) y  (\ref{eq:fim_19_2_6}) se tiene,

$$
2I=  \int_{0}^{\pi/2} \pi \sen^2 x \cos^2 x  \dd{x}  
$$

Resolviendo la integral se obtiene

$$
I=\frac{\pi^2}{32}
$$
 
\begin{LnxRptaBox}
	$$
 \int_{0}^{\pi} \frac{x \sen^{2n+2}(x) \cos^2 (x) }{  \sen^{2n}(x) +\cos^{2n}(x)} \dd{x}= \frac{\pi^2}{32}
	$$
\end{LnxRptaBox}

\solucion{Solución 3.b} 


$$
I= \int_{0}^{ \arctan(a/b) } \frac{b\sen x +a\cos x}{a\sen x+b\cos x} 
\dd{x}
$$

Dividimos entre $b$ el numerador y denominador

$$
I = \int_{0}^{ \arctan(a/b) } \frac{\sen x +\frac{a}{b}\cos x}{\frac{a}{b} \sen x+\cos x} \dd{x}   
$$

Hacemos sustitución $\frac{a}{b} =u $, entonces

$$
I= \int_{0}^{ \arctan(u) } \frac{\sen x +u\cos x}{u \sen x+\cos x}\dd{x}
$$

dividimos entre $\sen(x)$ el numerador y denominador

$$
I= \int_{0}^{ \arctan(u) } \frac{\tan x +u}{u \tan x+1} \dd{x}
$$

Usando sustitución $\tan(x) =m  \Rightarrow \dd{x} =\frac{\dd{m}}{m^2 +1} $
  
$$
 I= \ \int_{0}^{ u } \frac{m +u}{(um +1)(1+m ^2 )}\dd{m}
$$

Mediante fracciones parciales integramos
\[
\frac{m + u}{(u m + 1)(1 + m^2)} = \frac{\frac{u^3 - u}{u^2 + 1}}{u m + 1} + \frac{\frac{1 - u^2}{u^2 + 1} \cdot m}{m^2 + 1}+ \frac{\frac{2u}{u^2 + 1}}{m^2 + 1} 
\]

Entonces separamos en 3 integrales $I_1, I_2, I_3$

$
I= I_1+I_2+I_3
$ 

La primera integral $I_1$  es,

$$
I_1= \ \int_{0}^{ u } \left(\frac{u^2-1}{u^2+1} \right) \left(\frac{u}{mu+1} \right)  \, \mathrm{d} m 
$$

$$
I_1=\eval{ \left(\frac{u^2-1}{u^2+1} \right) \ln \vert um+1 \vert }_0^u = \left(\frac{u^2-1}{u^2+1} \right) \ln (u^2+1)
$$

La segunda integral $I_2$  es,

$$
I_2=\ \int_{0}^{ u }
\left( \frac{1-u^2}{2u^2+2}\right)\left(\frac{2m}{m^2+1} \right) \dd{m}
$$

$$
I_2 =\eval{ \frac{1-u^2}{2(u^2+1)} \ln (m^2+1)}_{0}^{u} = \frac{1-u^2}{2(u^2+1)} \ln (u^2+1)
$$
 
La tercera integral $I_3$  es,
 
$$
I_3= \int_{0}^{ u } \pqty{\frac{2u}{u^2+1}} \frac{1}{(m^2+1)} \dd{m}
$$

$$
I_3=\pqty{\frac{2u}{u^2+1}}\arctan(m)  \bigg \rvert_0^u=\frac{2u}{u^2+1} \arctan(u) 
$$

Sumando $\mathrm{I_1}+ \mathrm{I_2}+ \mathrm{I_3}=I$

$$
I= \frac{(u^2-1)}{2(u^2+1)} \ln (u^2+1)+\frac{2u}{u^2+1} \arctan(u)$$

donde $  u=\frac{a}{b} $, por lo tanto

\begin{LnxRptaBox}
	$$ 
	I= \frac{(a^2-b^2)}{2(a^2+b^2)} \ln \left(\frac{a^2+b^2}{b^2} \right)+\frac{2ab}{a^2+b^2} \arctan(\frac{a}{b})
	$$
\end{LnxRptaBox}

%%%%%%%%%%%%%%%%%%%%%%%%%%%%%%%%%%%%%%%%%%
\solucion{Solución 4} 

$$
f(x)=\ \int_{0}^{ x } e^{t^2} \, \mathrm{d}t \Rightarrow f(0)=0
$$

$$
f(1)=k=\ \int_{0}^{ 1 } e^{t^2}\dd{t}
$$

$$
f^{'}(x)=e^{x^2} 
$$


\solucion{Solución 4.a} 

$$ 
\mathrm{Int}= \ \int_{0}^{ k } \left[ \frac{y}{e^{(f^{-1} (y))^2}} -f^{-1} (y) \right]\, \mathrm{d} y   
$$

$$
f(f^{-1} (y))=\ \int_{0}^{ f^{-1} (y) } e^{t^2} \, \mathrm{d}t =y
$$

Tomando la primera derivada respecto a $y$, se tiene

$$
1=e^{(f^{-1} (y))^2} \left(f^{-1} (y) \right)^{'}  
\Rightarrow \frac{y}{e^{(f^{-1} (y))^2}} =y \left(f^{-1} (y) \right)^{'}
$$

$$
\mathrm{Int}= \ \int_{0}^{ k } \left[ y \left(f^{-1} (y) \right)^{'} -f^{-1} (y) \right]\, \mathrm{d} y  \hspace{.3 cm}  \hspace{.3 cm} 
$$

Hacemos sustitución $ m=f^{-1} (y)$ para continuar

$$
f^{'} (m) \, \mathrm{d} m= \mathrm{d}y \hspace{.3 cm};  \hspace{.3 cm} y=k \Rightarrow m=1 \hspace{.3 cm};  \hspace{.3 cm} y=0 \Rightarrow m=0
$$

$$
\mathrm{Int}= \ \int_{0}^{ 1 } \left[ f(m) \left(m \right)^{'} -m \right] f^{'} (m) \, \mathrm{d} m
$$

$$
\mathrm{Int}= \ \int_{0}^{ 1 }  f(m) f^{'} (m) \, \mathrm{d} m -\int_{0}^{ 1 } m e^{m^2} \, \mathrm{d} m 
$$

$$
\mathrm{Int}= \frac{f^2 (m)}{2} -\frac{e^{m^2}}{2} \bigg\rvert_0^1 
$$

\begin{LnxRptaBox}
$$ 
\mathrm{Int}= \frac{k^2 -e+1}{2}
$$
\end{LnxRptaBox}

\solucion{Solución 4.b}
 
$$
I_b= \ \int_{0}^{ 1 } \int_{0}^{ z } \int_{0}^{ y }  ze^{x^2+y^2}  \mathrm{d} x \,  \mathrm{d} y \,  \mathrm{d} z 
$$

Distribuimos las variables  

$$
I_b= \ \int_{0}^{ 1 } z \int_{0}^{ z } e^{y^2} \int_{0}^{ y }  e^{x^2}  \mathrm{d} x \,  \mathrm{d} y \, \mathrm{d} z     
$$

$$
\mathrm{I_b}= \ \int_{0}^{ 1 } z \int_{0}^{ z } e^{y^2} f(y) \,  \mathrm{d} y \,  \mathrm{d} z  =\int_0^1 z \int_0^z f^{'}(y)f(y)\, \mathrm{d} y \, \mathrm{d } z     
$$

$$ 
\mathrm{I_b}= \ \int_{0}^{ 1 } z \left[ \frac{f^2(y)}{2} \bigg \rvert_0^z \right]  \,  \mathrm{d} z  =\frac{1}{2}\int_0^1 z f^2(z)  \, \mathrm{d } z     
$$


Integramos por partes  $\displaystyle \int zf^2(z) \, \mathrm{d} z$
 \begin{itemize}
 	\item[] $z\, \mathrm{d} z=\mathrm{d} v\Rightarrow \frac{z^2}{2}=v$
 	\item[] $ f^2(z)=u \Rightarrow 2f^{'}(z) f(z) \, \mathrm{d} z =\mathrm{d}u$
 \end{itemize}

Reemplazando en $\int u \dd{v}=uv-\int v \dd{u}$
$$
2I_b= \int zf^2(z) \, \mathrm{d} z=\frac{z^2 f^2(z)}{2}- \int z^2 f^{'}(z) f(z) \, \mathrm{d} z 
$$

$$
2I_b=\ \int zf^2(z) \, \mathrm{d} z=\frac{z^2 f^2(z)}{2}- \int z f(z) z e^{z^2}  \, \mathrm{d} z 
$$ 

Integramos por partes, $\displaystyle I_c= \int z f(z) z e^{z^2} \, \mathrm{d} z$

 \begin{itemize}
	\item[] $z e^{z^2}\, \mathrm{d} z=\mathrm{d} m\Rightarrow \frac{ e^{z^2}}{2}=m$
	\item[] $zf(z)=n \Rightarrow (f(z)+zf^{'}(z))\, \mathrm{d} z =\mathrm{d}n$
\end{itemize}
 
$$
I_c=\ \int z f(z) z e^{z^2} \, \mathrm{d} z=\frac{ze^{z^2}f(z)}{2}-\int \frac{e^{z^2}}{2} \left(f(z)+zf^{'}(z) \right) \, \mathrm{d} z 
$$

$$
I_c=\frac{ze^{z^2}f(z)}{2}-\int \frac{f^{'}(z)f(z)}{2} \, \mathrm{d} z -\frac{1}{2} \int z e^{2z^2}  \, \mathrm{d} z 
$$

$$
I_c=\frac{ze^{z^2}f(z)}{2}- \frac{f^2(z)}{4}  -\frac{1}{8}  e^{2z^2} +C
$$

Reemplazando en $2I_b$

$$
\ \int zf^2(z) \, \mathrm{d} z=\frac{z^2 f^2(z)}{2}- \frac{ze^{z^2}f(z)}{2}+ \frac{f^2(z)}{4}  +\frac{1}{8}  e^{2z^2} +C
$$

$$ 
\mathrm{I_b}= \frac{1}{2}\int_0^1 z f^2(z)  \, \mathrm{d } z  =   
$$

$$ 
\mathrm{I_b}=   \frac{(2z^2+1) f^2(z)}{8}- \frac{ze^{z^2}f(z)}{4}  +\frac{1}{16} e^{2z^2} 
\bigg \rvert_0^1 
$$

Por lo tanto, tenemos

\begin{LnxRptaBox}
$$ 
\mathrm{I_b}=\frac{3k^2-2ek}{8}+\frac{e^2-1}{16}
$$
\end{LnxRptaBox}


}
